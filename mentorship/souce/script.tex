\documentclass[10pt,a4paper]{article}
\author{William Seymour (L2, Coventry)}
\date{UK, Ireland, and South Africa Judge Conference, December 2014 \\ Last updated \today}
\title{Mentorship Seminar}
\begin{document}
\maketitle
\clearpage
\tableofcontents
\section{Introduction}
Hello! My seminar today is on mentorship. 

First up, who am I? A very deep and philosophical question, which I don't have nearly enough time to go into here. What I can do is give you a bit of information about myself. I'm a computer science undergrad at the University of Warwick, where I've been mentoring various people as they take their first steps in programming. I also enjoy doing my bit for local judges and judge candidates. That said, I also spend a lot of my time \textit{being} mentored by other judges and by academic staff (or at least lectured at).

Right, mentorship.  Something we look for, but generally doesn't get taught, as such. Just so we're all on the same page it would be good to start off by asking, what \textit{is} the definition of mentorship? As always, google is your friend.

\begin{enumerate}
	\item ``an experienced and trusted adviser''
	\item ``an experienced person in a company or educational institution who trains and counsels new employees or students''
\end{enumerate}

Generally, mentorship differs from teaching in that broadly it is more focused on transferring experience than knowledge. In the context of judging, the obvious case where mentorship is required is certifying candidates - the path from L0 to L1, L1 to L2 etc. - but this isn't the only situation, as we'll see later. Events are also a key time for to engage in this, passing knowledge and experience on to other judges in a more practical context. It seems obvious that mentorship is a learning process, but not necessarily that it goes both ways. When done well, you'll find that you learn just as much as your mentees. This leads into our next question: can I mentor others even if I can't certify them? Yes. While you might not be the one to actually certify them, if they're at your local store, chances are you're the only judge they have regular face to face contact with. 

So make yourselves comfortable, and set your alarms for half an hour's time. We're going to start by going over the basics of mentoring someone before exploring how you might apply this to both local candidates and other judges at events.

\section{General Techniques}
Different people learn in different ways. Your candidate might know how they learn best. They might think they know how they learn best (which is more dangerous!). You'll often here that some people learn visually, others practically, and so on, but how do you use this to inform how you teach? For visual learners you could bring the CR and some cards up in front of them and go through a particular section. If audio is their thing, you might consider sending them to judge cast. The more practically minded might benefit from role playing scenarios (not \textit{that} sort of role playing). Eventually you'll want to have them do this at an event - doing is paramount. No matter how much they might try and convince you otherwise, people will quickly forget anything that they don't use. This is why you can key in the code for the office door without thinking about it but struggle to remember the password for that website you signed up for last month. Oh, you use the same password for everything? Well, yeah, about that.

You might be tempted to stop your candidate making mistakes - I mean, the whole point of this is to stop them making mistakes. But this is approaching the situation from completely the wrong direction. You want your candidate to make \textit{as many mistakes as possible}, because, with any luck, each mistake they make with you is one they won't make out on the front line by themselves. This has the added benefit of you being able to give them insight into why they made the mistake, and how they can recover (whatever it is, you've probably done it before).

\section{Mentorship of New Judges}
The typical scene, played out in game stores up and down the country. There you are at FNM, minding your own business. Just trying to figure out how Ertai's Meddling interacts with morph, when Jo comes up to you and asks ``how do I become a judge?'' Maybe you approached them to see if they'd considered becoming a judge. Regardless, where do you go from here?

Before you can do anything, before you can go anywhere, you have to know what they know, know what they don't know, and know what they think they know that `aint so. These broadly fall into two categories: hard skills, like rules and policy knowledge, and soft skills, such as how they deliver rulings, deal with awkward players and so on. To get a handle on their rules knowledge, ask them how they did on the last judge centre exam that they took. Maybe this would be a good time for them to go away and take the rules advisor test. Not only can this give you a general idea of where they are on the journey to level one, but it may give you some insights into the areas of the CR and MTR that they're not so good at. Of course, as with all judge centre exam content, no specifics please! Unfortunately there's no such exam to test for soft skills, that's, erm, why they're called soft skills after all. Luckily for you, an evening spent watching your judge candidate doing what they do best at FNM (not from a tree outside their bedroom window) will tell you all you need to know. Look at their demeanour, how they articulate rulings to players. In fact, put yourself in the player's shoes and think about how they came across. Then you should give them their shoes back, stealing is wrong. You could also ask the players, but that would require human interaction.

By now you should have an idea of what they need to work on. Find or come up with some scenarios to test their understanding. Framing these in terms of recent sets is usually best - use cards that they already know and understand. If you're lacking inspiration, take a look online. They will ask questions. Now, here comes the key. What should I do when my mentee asks me a question? Don't tell them the answer. Maybe I should clarify, please, tell them where they can \textit{find} the answer, but don't tell them \textit{what} the answer actually is. Of course, there are some exceptions to this - consult the building schematics is not an acceptable answer to ``where's the fire exit?''. So what exactly do I mean by this? Instead of this: 
\newline
\lbrack What happens if Erati's Meddling is cast on a morph spell?\rbrack
\newline
\lbrack Well, a copy of the morph spell is placed onto the stack. As in, a 2/2 nameless colourless creature with no text. It doesn't have a reverse face, because it's a copy.\rbrack
\newline
How about this:
\newline
\lbrack What happens if Erati's Meddling is cast on a morph spell?\rbrack
\newline
\lbrack
Which sections of the CR might be able to help you? Copying Objects might be a good place to start.\rbrack
\newline
This will really come into its own when they find questions on judge centre tests that they don't understand. This time they can't ask you for help, so it's up to them to consult the documents and arrive at an answer.

\begin{quotation}
	``Spoon feeding in the long run teaches us nothing but the shape of the spoon'' - E.M. Forster
\end{quotation}

OK, so you've been helping your candidate out with the areas they're not so strong at, test scores are up, and the sun is shining. What next? See them judge! That's what's next. Again, FNM is the ideal proving ground for fresh bloo- I mean new judges. Take a look at your notes from when you last observed them and look for improvement. If you think they're ready, \textit{now} you can think about certification. The last thing they're going to need is a passing score on a practise test. If your candidate is really as ready as you think they are, this should really come naturally.

\section{Mentorship at Events}
Now we've got the easy bit out the way - you expect to have to mentor someone who wants to become a judge - we can move on to the situations where you might not necessarily consider it as an option. I'm talking about mentoring during magic events.

Firstly, talk to as many people as possible. As an aside, this is something that can really make a difference to players during events as well. While not mentorship, the more interaction you have with new faces, the higher your retention levels are going to be.  New players to your store aren't going to come back if nobody talks to them. This concept applies to judging as well: having a bad experience at your first PTQ or GP means that you're less likely to pursue judging in that community, be it nationally or internationally.

Back on topic now, do your research. When you check the event forums beforehand, look out for people for whom this event is their first or second. Maybe they're looking to improve something specific. These are all perfect entry points for you to flex your mentorship muscles. The head judge talk is another great place to get inspiration here. Stop worrying about what you're going to say to everyone, and listen to the other judges. If all else fails, you could even ask them as a last resort.
 
When you have a spare five minutes during floor coverage, have a think about interesting scenarios related to their area of interest. Again, google is your friend. Engage with them on the area they want to improve in. When you're discussing examples focus not on the answer they give, but on the reasoning behind it. Even if they're right, play devil's advocate and encourage them to defend their point of view.

Maybe you have the opportunity to work more closely with them. If you believe that you have knowledge you can impart, ask to be on their team/their buddy. If you're not the one asking ``how do I do this'' you probably have knowledge to impart. A good starting point is to show them first. Tell them what you're doing, teach them the reasoning behind it, and then encourage them to have a go themselves. You should take notes (you should \textit{always} take notes) so that you can feed back to them. Always suggest ways they could improve. Letting them know what they did that they shouldn't do will make them good. Working out what they didn't do that they could be doing will make them \textit{great}.

\begin{quotation}
``Tell me and I forget, teach me and I may remember, involve me and I learn'' - Benjamin Franklin
\end{quotation}

Now, when I say watch them, I mean don't interfere. At all. Yes even that. Well, fine, if they're going to do \textit{that} then you might want to tap them on the shoulder and get them to reconsider. You want them to make as many mistakes as possible while you're watching. If you step in and hijack their mistakes, you help no one. Because you've been noting all this down, you can go over it with them afterwards. At this point, they might look to you for help or confirmation. Reassure them, by all means, but try not to actually give any input. Not only does it stop them making mistakes, like we just mentioned, but it also drains all of the confidence from the person they're currently helping. Even if you're showing them deck check techniques, wait for them to finish before giving advice and feedback.

\section{Conclusion}
As we begin to bring this meandering mentor-ship of insight and advice into the port of a well thought out conclusion (that was so forced), let's have a quick recap of the main points so far.

\begin{enumerate}
	\item Doing is paramount
	\item Don't answer your mentee's questions
	\item Certification is not the only opportunity for mentorship
\end{enumerate}

If you're interested in the slides or the script from this seminar, you can find it all at \textbf{mentorship.wseymour.co.uk}. An audio book edition is coming soon if you want to drift off to sleep in a more comfortable environment than the conference hall. I have some, well recommended reading isn't quite the right word ... articles that I'd like to suggest if you're interested in this. The first article I'd like to recommend is one by David Hibbs, which frames mentorship in the context of the qualities that are required of level three judges. Secondly, DLI wrote a piece on the feedback cycle, which as we've seen is an invaluable part of this process.

I hope I've got across to you the kinds of environments where you should consider mentoring people, and that by doing so you realise that it's likely far more relevant than you first thought. Remember, don't be afraid, don't be lazy, and you might just be able to teach someone else and yourself something new. This is what keeps the judge program running, and it's easy to pass on the favour. You know this stuff, so don't hide it.

Chances are, you're all as enthusiastic as you are (at least, when you're not in this seminar) because someone senior inspired you and helped you grow. Now it's time to pass it on. Thank you for giving me the opportunity to listen to the dulcet tones of my own voice for so long.

\section{Further Reading}
\begin{enumerate}
	\item Qualities of Regional Judges: Mentorship 
	\subitem [EN] blogs.magicjudges.org/articles/2013/09/03/l3-qualities-mentorship/
	
	\item The feedback process
	\subitem [EN] blogs.magicjudges.org/articles/2014/09/23/the-feedback-process/
	\subitem [FR] blogs.magicjudges.org/articles/2014/09/23/le-processus-de-feedback/
	\subitem [ES] blogs.magicjudges.org/articles/2014/09/23/el-proceso-de-feedback/
	\subitem [PT] blogs.magicjudges.org/articles/2014/09/23/processo-de-feedback/
\end{enumerate}
\end{document}